% !TEX program = xelatex

\documentclass{resume}
%\usepackage{zh_CN-Adobefonts_external} % Simplified Chinese Support using external fonts (./fonts/zh_CN-Adobe/)
%\usepackage{zh_CN-Adobefonts_internal} % Simplified Chinese Support using system fonts

\begin{document}
\pagenumbering{gobble} % suppress displaying page number

\name{Roshan Swain}

\basicInfo{
  \email{swainroshan001@gmail.com} \textperiodcentered\ 
  \phone{(+91) 9326457224} \textperiodcentered\ 
  \linkedin[snroshan]{https://www.linkedin.com/in/snroshan}}

\section{\faGraduationCap\ Education}
\datedsubsection{\textbf{National Institute of Technology (NIT)}, Agartala, India}{2018-- Present}
\textit{Undergraduate student} in Electrical Engineering (EE), Expected April 2022 \\
\textit {CGPA} : 8.69/10

\section{\faUsers\ Experience}
\datedsubsection{\textbf{Google Summer of Code '21} with org MLPACK}{May 2021 -- Present}
\role{Student Developer} 
\
\begin{itemize}
  \item Implementnig Mlpack library for designing GAN network of human faces generation. 
  \item Creating starter notebooks for showcasing potential use of library. 
\end{itemize}

\datedsubsection{\textbf{Mindkosh}}{March 2021 -- Present}
\role{Machine Learning Intern}
\
\begin{itemize}
  \item Developed Object Detection Model using ScaledYOLOv4 for bounding box annotations.
  \item Deployed Models on AWS for batch inferencing.
\end{itemize}

\section{\faGraduationCap\ Projects}
\datedsubsection{\textbf{Deep Neural Auto Encoder for Atlas Experiment}}{ Feb 2021}
Pytorch, FastAPI
\begin{itemize}
  \item Designed a deep neural autoencoder for compressing 4 variables data to 3 variables
  \item Optimized the model for a validataion loss of 2.8e-5 and residual of 2.98e-5
\end{itemize}

\datedsubsection{\textbf{German Traffic Sign Detection with Deep Learning}}{Dec 2020-- Jan 2021}
Pytorch
\begin{itemize}
  \item Modeled using ResNET architecture as a base for classifying 43 traffic signs
  \item Optimized the model for 99.8 percent accuracy
\end{itemize}

\section{\faGraduationCap\ Volunteer}
Open Source Contributions at Hacktoberfest 2020
\begin{itemize}
\item Maintained and Contributed to a pariticipating OpenSource repository  
\end{itemize}


% Reference Test
%\datedsubsection{\textbf{Paper Title\cite{zaharia2012resilient}}}{May. 2015}
%An xxx optimized for xxx\cite{verma2015large}
%\begin{itemize}
%  \item main contribution
%\end{itemize}

\section{\faCogs\ Skills}
\begin{itemize}[parsep=0.5ex]
  \item Programming Languages: C, C++, Python
  \item Framework \& Libraries: Pytorch, TensorFlow,ScikitLearn,Numpy,Pandas, Keras, Seaborn
  \item Database \& Cloud: SQL, AWS, GCP
\end{itemize}

% \section{\faHeartO\ Honors and Awards}
% \datedline{\textit Academic Excellence Award}{2015}
% Awarded for securing class \textit{Xth} in School

\section{\faInfo\ Miscellaneous}
\begin{itemize}[parsep=0.5ex] 
  \item GitHub: https://github.com/swaingotnochill
  \item Languages: English - Fluent, Hindi - Native Speaker
\end{itemize}

%% Reference
%\newpage
%\bibliographystyle{IEEETran}
%\bibliography{mycite}
\end{document}
